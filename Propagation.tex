\section{Graph-based name propagation}
\label{sec:graph}

The MOTIF team (IRISA/Inria-Rennes and PUC Minas) considered a graph-based approach where each node represents a ``speaking face''.

\subsection{Graph generation details}
\label{ssec:graph_gen}

A node is created for every ``speaking face'', namely each face track which temporally overlaps a speech segment by at least 60\%. If several speech segments overlap it, the face track is associated the one with the maximal overlap.

Edges between nodes can be weighted according to their audio or visual similarities. 

We compute the visual similarity $\sigma^V_{ij}$ as a cosine similarity between the OpenFace feature vectors $v_i$ and $v_j$ related to the face tracks of two nodes $N_i$ and $N_j$: $\sigma^V_{ij}=1/2+\frac{v_i\cdot v_j}{||v_i||*||v_j||}$, where $\cdot$ is the dot product and $||.||$ is the L2 norm.

The similarity $\sigma^A_{ij}$ between the speech segments of two nodes  $N_i$ and $N_j$ are computed as follows. Each speech segment is modeled with a 16-Gaussian mixture model (GMM) over Mel-frequency cepstral features. An Euclidean-based approximation of the KL2 divergence, noted $\delta^A_{ij}$, is then computed between the two GMMs~\cite{Ben}, and turned into a
similarity according to $\sigma^A_{ij}=\exp(\log{(\alpha)} \; \delta^A_{ij})$, where $\alpha = 0.25$.

\subsection{Name propagation}

TBA

\subsubsection{Random walk}

TBA

\subsubsection{Minimum spanning tree}

TBA

\endinput
